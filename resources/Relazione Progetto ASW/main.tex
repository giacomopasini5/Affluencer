\documentclass[a4paper]{report}
\usepackage[utf8]{inputenc}
\usepackage[italian]{babel}

\title{
    Affluencer \\
    \large Applicazioni e Servizi Web
}

\author{
    Daniele Gambaletta - 00000000 \{daniele.gambaletta@studio.unibo.it\} 
    \\
    Michele Durante - 00718472 \{michele.durante3@studio.unibo.it\}
    \\
    Giacomo Pasini - 00000000 \{giacomo.pasini5@studio.unibo.it\}
}

\date{Febbraio 2021}

\begin{document}

\maketitle

\pagebreak

\tableofcontents

\pagebreak

\chapter{Introduzione}

In questo capitolo verranno esposte le idee e le intenzioni che hanno portato alla realizzazione del progetto Affluencer. 

\section{Una nuova realtà}

Il 9 Marzo 2020 è stato l'inizio del primo lockdown in Italia, causato dalla corrente pandemia di Coronavirus.
%
Da quel momento sono state diverse le situazioni di paura e incertezza da parte degli imprenditori, soprattutto per chi non dispone di un'elevata leva economica.
%
L'impreparazione del governo e l'indebolimento delle strutture sanitarie ha fatto sì che molte piccole attività chiudessero, rassegnate di fronte allo scarso supporto economico e alle difficoltà di aderire alle procedure sanitarie richieste.
%
Quest'ultima parte ha creato non poche diatribe, che tutt'ora perdurano, tra gli imprenditori e il comitato tecnico scientifico: sebbene i locali venissero messi in sicurezza e controllati dagli apparati di polizia, l'introduzione del coprifuoco serale e il take-away come unica opzione per i ristoratori hanno portato gli imprenditori a protestare e ad attivarsi autonomamente, non curanti delle multe e dei potenziali rischi di contagio.

Una delle caratteristiche che il CTS...

\section{La nostra proposta}

Il progetto Affluencer si presenta come uno strumento in mano sia ai negozianti che ai privati cittadini. 
%
Tutti coloro che dispongono di un'attività possono registrarsi al servizio ed ottenere un misuratore di flusso, aggiornabile in tempo reale.
%
I privati protranno cercare i negozianti  direttamente da una mappa interattiva e valutare la situazione, scegliendo quando e come intraprendere il percorso più adatto.

\pagebreak

\chapter{Requisiti}

\chapter{Design}

\chapter{Tecnologie}

\chapter{Codice}

\chapter{Test}

\chapter{Deployment}

\chapter{Conclusioni}

\end{document}


